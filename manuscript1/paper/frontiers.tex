%%%%%%%%%%%%%%%%%%%%%%%%%%%%%%%%%%%%%%%%%%%%%%%%%%%%%%%%%%%%%%%%%%%%%%%%%%%%%%%%%%%%%%%%%%%%%%%%%%%%%%%%%%%%%%%%%%%%%%%%%%%%%%%%%%%%%%%%%%%%%%%%%%%%%%%%%%%
% This is just an example/guide for you to refer to when submitting manuscripts to Frontiers, it is not mandatory to use Frontiers .cls files nor frontiers.tex  %
% This will only generate the Manuscript, the final article will be typeset by Frontiers after acceptance.                                                 %
%                                                                                                                                                         %
% When submitting your files, remember to upload this *tex file, the pdf generated with it, the *bib file (if bibliography is not within the *tex) and all the figures.
%%%%%%%%%%%%%%%%%%%%%%%%%%%%%%%%%%%%%%%%%%%%%%%%%%%%%%%%%%%%%%%%%%%%%%%%%%%%%%%%%%%%%%%%%%%%%%%%%%%%%%%%%%%%%%%%%%%%%%%%%%%%%%%%%%%%%%%%%%%%%%%%%%%%%%%%%%%

%%% Version 3.1 Generated 2015/22/05 %%%
%%% You will need to have the following packages installed: datetime, fmtcount, etoolbox, fcprefix, which are normally inlcuded in WinEdt. %%%
%%% In http://www.ctan.org/ you can find the packages and how to install them, if necessary. %%%

\documentclass{frontiersSCNS} % for Science, Engineering and Humanities and Social Sciences articles
%\documentclass{frontiersHLTH} % for Health articles
%\documentclass{frontiersFPHY} % for Physics and Applied Mathematics and Statistics articles

%\setcitestyle{square}
\usepackage{url,hyperref,lineno,microtype}
\usepackage[onehalfspacing]{setspace}
\linenumbers


% Leave a blank line between paragraphs instead of using \\


\def\keyFont{\fontsize{8}{11}\helveticabold }
\def\firstAuthorLast{Sample {et~al.}} %use et al only if is more than 1 author
\def\Authors{First Author\,$^{1,*}$, Co-Author\,$^{2}$ and Co-Author\,$^2$}
% Affiliations should be keyed to the author's name with superscript numbers and be listed as follows: Laboratory, Institute, Department, Organization, City, State abbreviation (USA, Canada, Australia), and Country (without detailed address information such as city zip codes or street names).
% If one of the authors has a change of address, list the new address below the correspondence details using a superscript symbol and use the same symbol to indicate the author in the author list.
\def\Address{$^{1}$Laboratory X, Institute X, Department X, Organization X, City X , State XX (only USA, Canada and Australia), Country X \\
$^{2}$Laboratory X, Institute X, Department X, Organization X, City X , State XX (only USA, Canada and Australia), Country X  }
% The Corresponding Author should be marked with an asterisk
% Provide the exact contact address (this time including street name and city zip code) and email of the corresponding author
\def\corrAuthor{Corresponding Author}
\def\corrAddress{Laboratory X, Institute X, Department X, Organization X, Street X, City X , State XX (only USA, Canada and Australia), Zip Code, X Country X}
\def\corrEmail{email@uni.edu}




\begin{document}
\onecolumn
\firstpage{1}

\title[Running Title]{Network Analysis of Cropping Practices and Injury Profiles in Irrigated Rice Agroecosystems} 

\author[\firstAuthorLast]{\Authors} %This field will be automatically populated
\address{} %This field will be automatically populated
\correspondance{} %This field will be automatically populated

\extraAuth{}% If there are more than 1 corresponding author, comment this line and uncomment the next one.
%\extraAuth{corresponding Author2 \\ Laboratory X2, Institute X2, Department X2, Organization X2, Street X2, City X2 , State XX2 (only USA, Canada and Australia), Zip Code2, X2 Country X2, email2@uni2.edu}


\maketitle

%%%%%%%%%%%%%%%%%%%%%%%%%%%%%%%%%%%%%%%%%%%%%%%%%%%%%%%%%%%%%%%%%%%%%%%%%%%%%%%%%%%%%%%%%%%%%%%%%%%%%%%%%%%%%%%%%%%%%%%%%%%%%%%%%%%%%%%%%%%%%%%%%%%%%%%%%%%%%%%%%%%%%%%%%%%%%%%%%%%%%%%%%%%%%%%%%%%%%%%%%%%%%%%%%%%%%%%%%%%%%%%%%%%%%%%
%%% The sections below are for reference only.
%%%
%%% For Original Research Articles, Clinical Trial Articles, and Technology Reports the section headings should be those appropriate for your field and the research itself. It is recommended to organize your manuscript in the
%%% following sections or their equivalents for your field:
%%% Abstract, Introduction, Material and Methods, Results, and Discussion.
%%% Please note that the Material and Methods section can be placed in any of the following ways: before Results, before Discussion or after Discussion.
%%%
%%%For information about Clinical Trial Registration, please go to http://www.frontiersin.org/about/AuthorGuidelines#ClinicalTrialRegistration
%%%
%%% For Clinical Case Studies the following sections are mandatory: Abstract, Introduction, Background, Discussion, and Concluding Remarks.
%%%
%%% For all other article types there are no mandatory sections.
%%%%%%%%%%%%%%%%%%%%%%%%%%%%%%%%%%%%%%%%%%%%%%%%%%%%%%%%%%%%%%%%%%%%%%%%%%%%%%%%%%%%%%%%%%%%%%%%%%%%%%%%%%%%%%%%%%%%%%%%%%%%%%%%%%%%%%%%%%%%%%%%%%%%%%%%%%%%%%%%%%%%%%%%%%%%%%%%%%%%%%%%%%%%%%%%%%%%%%%%%%%%%%%%%%%%%%%%%%%%%%%%%%%%%%%

\begin{abstract}

%%% Leave the Abstract empty if your article falls under any of the following categories: Editorial Book Review, Commentary, Field Grand Challenge, Opinion or specialty Grand Challenge.
\section{}
%As a primary goal, the abstract should render the general significance and conceptual advance of the work clearly accessible to a broad readership. References should not be cited in the abstract.
For full guidelines regarding your manuscript please refer to \href{http://www.frontiersin.org/about/AuthorGuidelines}{Author Guidelines} \\ or \textbf{Table \ref{Tab:01}} for a summary according to article type.


\tiny
 \keyFont{ \section{Keywords:} Text Text Text Text Text Text Text Text } %All article types: you may provide up to 8 keywords; at least 5 are mandatory.
\end{abstract}

\section{Introduction}

% use only four paragraphs for the introduction.
% The introduction to the useful of the survey data

% How do the surveys themselves lead to better management? You're missing an important step here.
The use of in-field surveys is a useful tool to develop ground-truth databases that allow one identify actual constraints due to pests in an agricultural productions system. These sorts of databases provide an overview of the complex relationships between the crop, its management, pest injuries, yields. Understanding theses relationships may lead to better management, and guide researchers the new research hypotheses \citep{mew:2004kh, Savary:2006to}. 

%===state the problems===
% why are there "'" throughout? Use "'" instead. I already did a search/replace, but please do not do this again. I've pointed this out more than 10X already, yet you continue. Attention to detail is one of your biggest weaknesses. With this issue, I'm not even sure how it occurs, but since it does, you need to be watching it.

% I've fixed your punctuation and capitalization here, please pay attention
% I don't understand the last sentence. You need to clarify what you mean here
Several previous studies \citep{Savary:2000char, savary2000quanti, savary2005multiple, dong2010characterization, Reddy:2011hl} involved surveys that have been used to identify relationships in an individual production situation (a set of factors that determine agricultural production) and the injury profiles (combination of disease and pest injuries that may occur in a given farmer's field) using nonparametric multivariate analysis such as cluster analysis, correspondence analysis, multiple correspondence analysis. Performing correspondence analysis \citep{savary1997new}, they characterized the relationships between categorized levels of variables: actual yield, production situations, and injuries profiles. Their results led to the conclusions that observed injuries profiles were strongly associated with production situations and the level of actual yields.

% the useful of network analysis 
The components of production situation and injury profiles are biologically related. For example, the excess amount of fertilizers applied in the rice files increases the susceptibility of rice to blast and directed seeded flooded rice fields with high seedling rate is the favorable condition for sheath blight \citep{ouricedisease}. The relationships will be more complex when the number of their components increased. A way to systemically model and intuitively interpret such relationships is the depiction as a graph or network. This approach has been widely used and proven very useful in biological studies \citep{Lefebvre:2011fo}. Networks typically consist of nodes, usually representing components, while links between the nodes depict their interactions \citep{PROULX:2005hx}. A correlation network is a type of network in which two nodes are connected if their respective correlation lies above a certain threshold. The construction of this network is obtained from pairwise correlation methods \citep{Toubiana:2013cv}. By using appropriate correlation measure, correlation networks can capture biologically meaningful relationships, and discover valuable information in crop health surveys.

% reword your first sentence for clarity
The main objective of this study is to apply network theory to the rice crop health survey data with the proposed methods for network construction. Selecting the suitable measure is important because the method should be able to capture the relationships with true concordance often determined the type and amount of knowledge we can gain from survey data, moreover it will affect the topological structure of network (the patterns of pairwise relationships between variables).
                                                                                                                                                                                                                                                                                                 %have limited prior knowledge (positive relation and negative relation) for comparing the efficiency of different association methods in discovering true functionally associated variables.

% it is NOT KANDELL. 
% it is not CONSTRAINS (that is a verb) you are using a noun here.
%The main aim of this article is to evaluate correlation methods including Pearson, Spearman, Kendall, Biweight to associate the components of cropping practices and the components of injuries profiles. Furthermore, we applied network theory and model to illustrate the pairwise relationships. Thus, we hope to provide the necessary elements for a better comprehension of the methods and also the choice of a suitable dependence test method based on practical constrains and goals.


%\begin{methods}
\section{Material \& Methods}

% always use "an" not "a" before a word starting with a vowel or soft consonant
% revise your first sentence for clarity
% revise your last sentence for clarity

In the first step of the methods, we compute partial correlation coefficients between each edge and a behavioral measure of interest independently, by taking other behavioral measures as covariates. Either Pearson or Spearman correlation coefficients can be employed, though we employed Spearman rank correlation coefficient since the distribution of the brain connectivity data is unknown and often distribution not satisfy the normality condition. (copy from the)

Next, we inferred association network from surveys comprising five countries (India, Indonesia, Philippines, Thailand, and Vietnam), 420 lowland farmers' fields. We determine the correlation patterns among the incidence of injuries caused by animal pests and diseases and the cropping practices, potentially indicative of their occurrence relations. We then constructed the network from these pairwise correlations. 

\subsubsection*{Survey datasets}
Crop health survey data were collected through surveys comprising 420 farmers' fields from 2010 to 2012 for wet and dry seasons in different production environments across South and South East Asia. The survey protocol described in the IRRI publication, ``A survey portfolio to characterize yield-reducing factors in rice'', was used for data collection \citep{Savarysurvey2009}. The variables collected included patterns of cropping practices, crop growth measurement and crop management status assessments, measurements of levels of injuries caused by pests, and direct measurements of actual yields from crop cuts. The data collected can be classified into three groups: cropping practices, injuries, and actual yield measurements.


\subsection*{Evaluation}

\subsubsection*{One: Data exploration}
There are three main properties to be determined before deciding the appropriate correlation measure for use in constructing the network. 

% again, stop with superlatives, please!
% You need to expand on your last sentence, I'm unclear what you're trying to say here.
\paragraph{Check data distribution} This test can be achieved by significance test and visual methods. Each variable in survey dataset was tested normality using the Shapiro-Wilk test \citep{ghasemi2012normality}. The Shapiro-Wilk test is based on the correlation between the data and the corresponding normal scores.

$H_{0}$: sample distribution is normal.

$H_{a}$: sample distribution is not normal.

% population or sample? Your surveys are a sample.
Thus, if the $p$-value is less than the chosen alpha level, the null hypothesis is rejected and there is evidence that the data tested are not from a normally distributed population. In other words, the data are not normal. On the contrary, if the $p$-value is greater than the chosen alpha level, then the null hypothesis that the data came from a normally distributed sample   cannot be rejected. However, for small sample sizes, normality tests have little power to reject the null hypothesis, so a QQ (quantile--quantile plot) plot and the frequency distribution (histogram) are required for verification, in addition, to check normality visually.

% there is no need to "download" the stats package from CRAN. It's a part of the R base installation. Revise.
The \texttt{R} function for Shapiro-Wilk Normality test is \texttt{shapiro.test} (package stats), whihc is \citep{R:2014}.


%====================
\subsection*{Network Construction}
\subsubsection*{Correlation network construction}

% Network analysis was performed using the ----. More . Major practical steps are described as follows. First, an RA matrix, a matrix of soil variable and OTU annotation file were prepared in the formats as guided in the pipline. Second, the RA matrix was submitted for network construction. Using default settings, a cutoff value (similarity theshold) for the similarity matrix was automatically generated. A link between a pair of OTUs is assigned when the correaltion beteen their RAs exceeds this threshold vlue. Third, calculations on global network properties, the individual nodes' centrality, and the module seperation and modularity" were oerformed. A module (or a cluster) is a group of nodes more densely connected to each other than to node outside

The matrix can be viewed as an adjacency matrix of a weighted network. The matrix contains the correlation coefficient between each node (i.e., the variable). Thus, the matrix can be thought of as the population average of the network structure. Because we are looking at several specific links, we control for multiple testing by controlling the False Discovery Rate (FDR method) at 5\%. The generated network structure can be visualized through the \texttt{R} package qgraph \citep{qgraph}. Only connections that surpass the significance threshold are shown in the visual representation. 

\subsection*{Evaluating Network properties}

To evaluate the topological properties of both the interaction  and the co-occurrence network, we used the package igraph and qgraph in The R environment. Particularly we were interested in properties potentially relevant for community roles and functioning as previously hypothesized in and reference therein m theres are :

\begin{itemize}
\item Mean degree <k>: the degree of a node counts the number of edges it has. The mean degree of nodes calculate over all  nodes in the network
\item Degree distribution: the frequency of node vs. their (increasing) degree.
\ item Average shortest path length,l.: the shortage path between any two nodes is the single path with fewest links between them. Alternative paths are feasible. The average shortest path length is the mean over all shortest oaths between any two nodes in the network.
\item Mean clustering coefficients <c>: a cluster of nodes  a triangle of nodes. The clustering coefficient calculates the fraction of observed vs possible triangles for each mode. The mean is subsequently determined from all nodes in the network,
\item Betweenness centrality <CB>
\item Closeness centrality <CC>
\end{itemize}

Important information about a network can be gained by analyzing its global structure, for example by looking at the relative centrality of different nodes. In a centrality analysis, nodes are ordered in terms of the degree to which they occupy a central place in the network. Global descriptors of the modules were obtained using package qgraph in \texttt{R}. The neighborhood of a given node $n$ is the set of its neighbors. The connectivity is the size of its neighborhood. The average number of neighbors indicates the average connectivity of a node in the network. A normalized version of this parameter is the network density. Density ranges between 0 and 1. It shows how densely the network is populated with edges. A network, which contains no edges and solely isolated nodes has a density of 0. 

In correlation (undirected) networks, the clustering coefficient is the number of connected pairs between all neighbors of the network. The clustering coefficient of a node is always a number between 0 and 1. The network clustering coefficient is the average of the clustering coefficients for all nodes in the network. Nodes with less than two neighbors are assumed to have a clustering coefficient of 0. We then determined network centralities on the modules obtained from network analysis. Centralities were assessed using qgraph package in \texttt{R}. We calculated Degree centrality and Betweenness centrality.

\begin{table}[!t]
\textbf{\refstepcounter{table}\label{Tab:01} Table \arabic{table}.}{ Maximum size of the Manuscript }

\processtable{ }
{\begin{tabular}{lllll}\toprule
 & Abstract max. legth (incl. spaces) & Figures or tables & Manuscript max. length \\\midrule
Clinical Case Study & & & &\\
Clinical Trial & & & &\\
Hypothesis and Theory & & & &\\
Methods & 2000 characters  & 15 & 12000 words \\
Original Research & & & &\\
Review & & & &\\
Technology Report & & & &\\\midrule
Focused Review & 2000 characters & 5 & 5000 words \\\midrule
CPC &  1250 characters& 6 & 2500 words  \\\midrule
Perspective & 1250 characters & 2 & 3000 words  \\
Mini Review & & & &\\\midrule
Data Report & None & 2 & 3000 words\\\midrule
Classification & 1250 characters & 10 & 2000 words \\\midrule
Editorial & None & None & 1000 words  \\\midrule
Frontiers Commentary  & & &\\
General Commentary & None & 1 & 1000 words\\
Book review & & & \\\midrule
Opinion   & & &\\
Specialty Grand Challenge & None & 1 & 2000 words\\
Field Grand Challenge & & & &\\\botrule
\end{tabular}}{}
\end{table}

Please note that large tables covering several pages cannot be included in the final PDF for formatting reasons. These tables will be published as supplementary material on the online article abstract page at the time of acceptance. The author will notified during the typesetting of the final article if this is the case. A link in the final PDF will direct to the online material.

\subsection{Original Research Articles, Clinical Trial Articles, and Technology Reports}

For Original Research Articles, Clinical Trial Articles, and Technology Reports the section headings should be those appropriate for your field and the research itself. It is recommended to organize your manuscript in the following sections or their equivalents for your field:

\begin{itemize}
%for bulleted list, use itemize
\item Introduction: Succinct, with no subheadings.
\item Materials and Methods: This section may be divided by subheadings. This section should contain sufficient detail so that when read in conjunction with cited references, all procedures can be repeated.
\item Results: This section may be divided by subheadings. Footnotes should not be used and have to be transferred into the main text.
\item Discussion: This section may be divided by subheadings. Discussions should cover the key findings of the study: discuss any prior art related to the subject so to place the novelty of the discovery in the appropriate context; discuss the potential short-comings and limitations on their interpretations; discuss their integration into the current understanding of the problem and how this advances the current views; speculate on the future direction of the research and freely postulate theories that could be tested in the future.
\end{itemize}

Please note that the Material and Methods section can be placed in any of the following ways: before Results, before Discussion or after Discussion.

\subsection{Clinical Case Studies}

For Clinical Case Studies the following sections are mandatory:

\begin{itemize}
%for bulleted list, use itemize
\item Introduction: Include symptoms at presentation, physical exams and lab results.
\item Background: This section may be divided by subheadings. Include history and review of similar cases.
\item Results: This section may be divided by subheadings. Include diagnosis and treatment.
\item Concluding Remarks
\end{itemize}

%\end{methods}



\section{Results}

\subsection{Figures}
Frontiers requires figures to be submitted individually, in the same order as they are referred to in the manuscript. Figures will then be automatically embedded at the bottom of the submitted manuscript. Kindly ensure that each table and figure is mentioned in the text and in numerical order. Permission must be obtained for use of copyrighted material from other sources (including the web). Please note that it is compulsory to follow figure instructions. Figures which are not according to the guidelines will cause substantial delay during the production process.


\begin{table}[!t]
\textbf{\refstepcounter{table}\label{Tab:02} Table \arabic{table}.}{ Resolution Requirements for the figures}

\processtable{}
{\begin{tabular}{lllll}\toprule
Image Type & Description & Format & Color Mode & Resolution\\\midrule
Line Art & An image composed of lines and text,  & TIFF, JPEG & RGB, Bitmap & 900 - 1200 dpi\\
           & which does not contain tonal or shaded areas.& & &\\
           Halftone & A continuous tone photograph, which contains no text. & TIFF, EPS, JPEG & RGB, Grayscale & 300 dpi\\
Combination & Image contains halftone + text or line art elements. & TIFF, JPEG & RGB,Grayscale & 600 - 900 dpi\\\botrule
\end{tabular}}{}
\end{table}

\textbf{Table \ref{Tab:02}} shows the resolution requirements for the figures. The figures must be legible:
\begin{enumerate}
\item The smallest visible text is no less than 8 points in height, when viewed at actual size.
\item Solid lines are not broken up.
\item Image areas are not pixelated or stair stepped.
\item Text is legible and of high quality.
\item Any lines in the graphic are no smaller than 2 points width.
\item The actual size of the figure must be of at least 8.5 cm.
\end{enumerate}

\subsection{Nomenclature}
\begin{itemize}
\item The use of abbreviations should be kept to a minimum. Non-standard abbreviations should be avoided unless they appear at least four times, and defined upon first use in the main text. Consider also giving a list of non-standard abbreviations at the end, immediately before the Acknowledgments.
\item Gene symbols should be italicized; protein products are not italicized.
\item Chemical compounds and biomolecules should be referred to using systematic nomenclature, preferably using the recommendations by IUPAC.
\item We encourage the use of Standard International Units in all manuscripts.
\item To take part in the Resource Identification Initiative, please cite antibodies, genetically modified organisms, software tools, data, databases and services using the corresponding catalog number and RRID in your current manuscript. For more information about the project and for steps on how to search for an RRID, please click \href{http://www.frontiersin.org/files/pdf/letter_to_author.pdf}{here}.
\end{itemize}

\begin{equation}
\sum x+ y =Z\label{eq:01}
\end{equation}

\section{Discussion}

Text Text Text Text Text Text  Text Text Text Text Text Text Text Text Text  Text Text Text Text Text Text Text Text Text Text.
Additional Requirements:
\subsection{Corrections}

If you need to communicate important changes to a published article please submit a General Commentary. Submit the article with the title “Corrigendum: Original Title of Article”.

\subsection{Commentaries on Articles}

At the beginning of your Commentary, please provide the citation of the article commented on. Rebuttals may be submitted in response to Commentaries; our limit in place is one commentary and one response. Rebuttals should also be submitted as General Commentary articles.

\subsection{Human Search and Animal Research}

All experiments on live vertebrates or higher invertebrates must be performed in accordance with relevant institutional and national guidelines and regulations. In the manuscript, authors must identify the committee approving the experiments and must confirm that all experiments conform to the relevant regulatory standards. For manuscripts reporting experiments on human subjects, authors must identify the committee approving the experiments and must also include a statement confirming that informed consent was obtained from all subjects. In Original Research Articles and Clinical Trial Articles these statements should appear in the Materials and Methods section.

\subsection{Clinical Trial Registration}

Clinical trials should be registered in a public trials registry in order to become the object of a publication at Frontiers. Trials must be registered at or before the start of patient enrollment. A clinical trial is defined as "any research study that prospectively assigns human participants or groups of humans to one or more health-related interventions to evaluate the effects on health outcomes."(\href{www.who.int/ictrp/en}{www.who.int/ictrp/en}). A list of acceptable registries can be found at \href{www.who.int/ictrp/en}{www.who.int/ictrp/en} and \href{www.icmje.org}{www.icmje.org}.

\subsection{Inclusion of Proteomics Data}

Authors should provide relevant information relating to how the peptide/protein matches were undertaken, including methods used to process and analyze data, false discovery rates (FDR) for large-scale studies and threshold or cut-off rates for peptide and protein matches. Further information could include software used, mass spectrometer type, sequence database and version, number of sequences in database, processing methods, mass tolerances used for matching, variable/fixed modifications, allowable missed cleavages, etc.

Authors should provide as supplementary material information used to identify proteins and/or peptides. This should include information such as accession numbers, observed mass (m/z), charge, delta mass, matched mass, peptide/protein scores, peptide modification, miscleavages, peptide sequence, match rank, matched species (for cross species matching), number of peptide matches, ambiguous protein/peptide matches should be indicated, etc.
For quantitative proteomics analyses authors should provide information to justify the statistical significance including biological replicates, statistical methods, estimates of uncertainty and the methods used for calculating error.

For peptide matches with biologically relevant post-translational modifications (PTM) and for any protein match that has occurred using a single mass spectrum, authors should include this information as raw data, annotated spectra or submit data to an online repository (recommended option).
Authors are encouraged to submit raw or matched data and 2-DE images to public proteomics repositories. Submission codes and/or links to data should be provided within the manuscript.

\subsection{Data Sharing}

Frontiers supports the policy of data sharing, and authors are advised to make freely available any materials and information described in their article, and any data relevant to the article (while not compromising confidentiality in the context of human-subject research) that may be reasonably requested by others for the purpose of academic and non-commercial research. In regards to deposition of data and data sharing through databases, Frontiers urges authors to comply with the current best practices within their discipline.

\section*{Disclosure/Conflict-of-Interest Statement}
%Frontiers follows the recommendations by the International Committee of Medical Journal Editors (http://www.icmje.org/ethical_4conflicts.html) which require that all financial, commercial or other relationships that might be perceived by the academic community as representing a potential conflict of interest must be disclosed. If no such relationship exists, authors will be asked to declare that the research was conducted in the absence of any commercial or financial relationships that could be construed as a potential conflict of interest. When disclosing the potential conflict of interest, the authors need to address the following points:
%•	Did you or your institution at any time receive payment or services from a third party for any aspect of the submitted work?
%•	Please declare financial relationships with entities that could be perceived to influence, or that give the appearance of potentially influencing, what you wrote in the submitted work.
%•	Please declare patents and copyrights, whether pending, issued, licensed and/or receiving royalties relevant to the work.
%•	Please state other relationships or activities that readers could perceive to have influenced, or that give the appearance of potentially influencing, what you wrote in the submitted work.

The authors declare that the research was conducted in the absence of any commercial or financial relationships that could be construed as a potential conflict of interest.

\section*{Author Contributions}
%When determining authorship the following criteria should be observed:
%•	Substantial contributions to the conception or design of the work; or the acquisition, analysis, or interpretation of data for the work; AND
%•	Drafting the work or revising it critically for important intellectual content; AND
%•	Final approval of the version to be published ; AND
%•	Agreement to be accountable for all aspects of the work in ensuring that questions related to the accuracy or integrity of any part of the work are appropriately investigated and resolved.
%Contributors who meet fewer than all 4 of the above criteria for authorship should not be listed as authors, but they should be acknowledged. (http://www.icmje.org/roles_a.html)

The statement about the authors and contributors can be up to several sentences long, describing the tasks of individual authors referred to by their initials and should be included at the end of the manuscript before the References section.


\section*{Acknowledgments}
Text Text Text Text Text Text  Text Text Text Text Text Text Text Text  Text Text Text Text Text Text Text Text Text  Text Text Text. Text Text Text Text Text Text  Text Text Text Text Text Text Text Text  Text Text Text Text Text Text Text Text Text  Text Text Text. 


\textit{Funding\textcolon} Text Text Text Text Text Text  Text Text.

\section*{Supplemental Data}
Supplementary Material should be uploaded separately on submission, if there are Supplementary Figures, please include the caption in the same file as the figure. LaTeX Supplementary Material templates can be found in the Frontiers LaTeX folder

Text Text Text Text Text Text  Text Text Text Text Text Text Text Text  Text Text Text Text Text Text Text Text Text  Text Text Text.


\bibliographystyle{frontiersinSCNS_ENG_HUMS} % for Science, Engineering and Humanities and Social Sciences articles, for Humanities and Social Sciences articles please include page numbers in the in-text citations
%\bibliographystyle{frontiersinHLTH&FPHY} % for Health and Physics articles
\bibliography{test}

%%% Upload the *bib file along with the *tex file and PDF on submission if the bibliography is not in the main *tex file

\section*{Figures}

%%% Use this if adding the figures directly in the mansucript, if so, please remember to also upload the files when submitting your article
%%% There is no need for adding the file termination, as long as you indicate where the file is saved. In the examples below the files (logo1.jpg and logo2.eps) are in the Frontiers LaTeX folder
%%% If using *.tif files convert them to .jpg or .png

\begin{figure}[h!]
\begin{center}
\includegraphics[width=10cm]{logo1}% This is a *.jpg file
\end{center}
 \textbf{\refstepcounter{figure}\label{fig:01} Figure \arabic{figure}.}{ Enter the caption for your figure here.  Repeat as  necessary for each of your figures }
\end{figure}

%\begin{figure}
%\begin{center}
%\includegraphics[width=10cm]{logo2}% This is an *.eps file
%\end{center}
%\textbf{\refstepcounter{figure}\label{fig:02} Figure \arabic{figure}.}{ Enter the caption for your figure here.  Repeat as  necessary for each of your figures }
%\end{figure}

%%% If you don't add the figures in the LaTeX files, please upload them when submitting the article.

%%% Frontiers will add the figures at the end of the provisional pdf automatically %%%

%%% The use of LaTeX coding to draw Diagrams/Figures/Structures should be avoided. They should be external callouts including graphics.

\end{document}
